\documentclass[a4paper,10pt]{article}

\usepackage[a4paper,left=1.0cm,right=1.0cm,top=2.0cm]{geometry}

\usepackage[utf8]{inputenc}
\usepackage{listings}
\usepackage{graphicx}
\usepackage[pdftex]{hyperref}
\usepackage{color}
\usepackage{textcomp}
\usepackage{float}

\definecolor{listinggray}{gray}{0.9}
\definecolor{lbcolor}{rgb}{0.9,0.9,0.9}

\lstset{
	backgroundcolor=\color{lbcolor},
	tabsize=2,
	rulecolor=,
	language=C,
        basicstyle=\scriptsize,
        upquote=true,
        aboveskip={1.5\baselineskip},
        columns=fixed,
        showstringspaces=false,
        extendedchars=true,
        breaklines=true,
        prebreak = \raisebox{0ex}[0ex][0ex]{\ensuremath{\hookleftarrow}},
        frame=single,
        showtabs=false,
        showspaces=false,
        showstringspaces=false,
        identifierstyle=\ttfamily,
        keywordstyle=\color[rgb]{0,0,1},
        commentstyle=\color[rgb]{0.133,0.545,0.133},
        stringstyle=\color[rgb]{0.627,0.126,0.941},
}

%opening
\title{4º Trabalho de Compiladores - Interpretador}
\author{André Siviero e Juan França}


\begin{document}

\maketitle

\begin{abstract}
Documentação referente à quarta parte do trabalho de compiladores, que consiste em um interpretador.
\end{abstract}

\section{Introdução}
O objetivo deste trabalho é estabelecer um interpretador para a linguagem PC (Poor C). Para construir esta parte, utiliza-se os três analisadores previamente construídos, ou seja,
os trabalhos anteriores foram incorporados nesta parte de tal forma que executar um programa checando erros sintáticos e semânticos da linguagem PC, exatamente como um interpretador faz.

\section{Menu}

Para facilitar a execução do código é criado um menu contendo as seguintes opções:

\begin{enumerate}
 \item Compilar um programa: construir a árvore de execução, referenciada no código por cmdList.
 \item Imprimir Árvore de Execução: basta imprimir cada nó da cmdList vasculhando também as árvores internas do nó
 \item Executar Programa: basta executar a cmdList. 
 \item Listar programas compilados
 \item Listar programas no diretório corrente(que podem ser compilados)
 \item Sair
\end{enumerate}

\section{Principais Estruturas}

Para criar e executar a árvore de execução é necessária a utilização de diversas estruturas. Cada estrutura segue a sua respectiva regra construída no segundo trabalho.
Cada estrutura possui a sua função de executar, normalmente, nomeada de \textbf{execute<estrutura>}. Por exemplo, para o fator, temos executeFator.

Segue abaixo a explicação sobre cada uma delas:

\subsection{Fator}
Esta é a mais baixa ordem da árvore. Pode ser inteiro, float, char, variável ou uma chamada de função. Sua estrutura é a seguinte:

\begin{lstlisting}
struct {
	int tipo;
	void *valor;
	list parametros;
} typedef s_fator;
\end{lstlisting}

A lista de parâmetros na chamada de função guarda os parâmetros passados pela função. É utilizada também para as atribuições ++/-- a variáveis. Um outro caso em que é utilizada,
é para explicitar que uma variável está com o valor negativo. 
O \textbf{tipo} serve para identificar qual typecast deve ser feito para capturar o valor correspondente a variável.\\
Esta manipulação é feita na função de execução. O caso de inteiro, char e float basta retornar o s_fator correspondente. Porém, para a chamada de função e variável existem
considerações.
\subsection{Termo}

O termo fica um nível acima na árvore do Fator. É composta apenas por um sinal de operação que pode ser '*','/' e '\%'. Realiza operações sobre os fatores que estão abaixo dele na árvore.

\begin{lstlisting}
 struct {
	char op;
} typedef s_termo
\end{lstlisting}

\subsection{EXP}
Basicamente, igual ao termo, porém os operadores são outros: '+','-'. Isto é feito para resolver o caso da precedência de operadores. Como a árvore é executada de baixo para cima as operações de multiplicação, divisão e mod serão executadas primeiro do que as de soma e subtração.

\begin{lstlisting}
struct {
	char op;
} typedef s_exp;
\end{lstlisting}

\subsection{U\_EXP}

Este é o caso em que os operadores: '>', '<', '>=' e '<=' foram inseridos. Assim pode-se comparar duas expressões e avaliar o valor.

\begin{lstlisting}
struct {
	char op[3];
} typedef s_u_exp;
\end{lstlisting}

\subsection{U\_Exp\_List}

É o conjunto de u\_exp e contém os operadores: '||', '\&\&' e '!=' para avaliar as u\_exp.

\begin{lstlisting}
struct {
	char op[2];
} typedef s_u_exp_list;
\end{lstlisting}

\subsection{Variable}

Estrutura reutilizada do terceiro trabalho para tratar as variáveis presentes no código. A sua estrutura é a seguinte:

\begin{lstlisting}
struct  {
  char nome[30];
  void *valor;
  int tipo;
  char escopo[30];
  int lineDeclared;
  int used;
} typedef s_variavel;
\end{lstlisting}

\subsection{Function}

Estrutura reutilizada do terceiro trabalho para tratar as funções presentes no código. A sua estrutura é a seguinte:

\begin{lstlisting}
struct  {
  char nome[30];
  int aridade;
  int tipo_retorno;
  list parametros;
  list parNames;
  list cmdList;
} typedef s_funcao;
\end{lstlisting}
\subsection{Atrib}

Referente as atribuições da linguagem C. 

\begin{lstlisting}
struct {
	char op[3];
	char varname[50];
	NODETREEPTR toatrib;
	char *stringToAtrib;
} typedef s_atrib;
\end{lstlisting}

Composta por um operador que pode ser '=','+=','-=','*=' e '/='. O nome da variável que receberá o valor do resultado da avaliação do nó toatrib. Foi criado o
stringToAtrib o tipo string de dados. 

\subsection{Conditional}

Correspondente aos comandos de seleção: if e else. A sua estrutura é a seguinte:

\begin{lstlisting}
struct {
	NODETREEPTR condition;
	list commandList;
	list elseCommandList;
} typedef s_conditional;
\end{lstlisting}

Composta pelo nó da condição e duas listas: execução do if e execução do else. 

\subsection{Loop}

Correspondente aos loops permitidos na linguagem PC: for, while e do-while. A sua estrutura é a seguinte:

\begin{lstlisting}
struct {
	NODETREEPTR condition;
	list commandList;
	list atribList;
	list incrementList;
	int tipo;
} typedef s_loop;
\end{lstlisting}

Os comandos de repetição tem a condição e três listas: a de comandos, a de atribuição e a de incremento. O comando FOR por exemplo seria da forma:

\begin{lstlisting}
for ( atribList; condition; incrementList){
  commandList
}
\end{lstlisting}

\subsection{Switch}

Para o switch utiliza-se a estrutura switch block que consiste em cada case referente ao respectivo switch. A lista de comandos é formada pelos ssb que possivelmente serão executados.
Cada switch possui o valor de check que será verificado com o valor do check do ssb. Como estabelicido no segundo trabalho, o switch deve obrigatoriamente ter um default para funcionar.

\begin{lstlisting}
struct switch_block{

    void* condition;
    list commands;
} typedef ssb;

struct {

	list commandList;
	char* check_value;
    char* check_value_s;

} typedef s_switch;
\end{lstlisting}

\subsection{Tree}
Corresponde a árvore genérica utilizada para . Cada nó corresponde a uma das estruturas acima mencionadas.
\begin{lstlisting}
struct nodeTree {
	void* element;
	struct nodeTree* next;
	list children;
	int tipoNodeTree;
};
\end{lstlisting}

\section{cmdList}

Esta é a lista principal do programa. É composta por nós de árvore genéricos.

\section{Exemplos}
Nos exemplos, exploramos diversas situações para os diversos tipos de usuários. Foram criadas 10 situações, 5 situações corretas e 5 
situações erradas.
\end{document}