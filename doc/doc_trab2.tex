\documentclass[a4paper,10pt]{article}
\usepackage[utf8x]{inputenc}

%opening
\title{2º Trabalho de Compiladores - Analisador Sintático}
\author{André Siviero e Juan França}

\begin{document}

\maketitle

\begin{abstract}
Documentação referente à segunda parte do trabalho de compiladores, que consiste em um analisador sintático.
\end{abstract}

\section{Introdução}
O objetivo deste trabalho é estabelecer um analisador sintático para a linguagem PC (Poor C). O analisador sintático é a segunda parte de um compilador,
e sua função é analisar a sintaxe dos elementos contidos em um arquivo de entrada (código fonte). Assim, nesta segunda etapa, busca-se verificar se o arquivo de entrada
está dentro dos padrões da linguagem Poor C, que será definida mais abaixo.

Utilizando a ferramenta Bison para gerar um analisador sintático a tarefa foi facilitada. Nesta documentação, são descritas as regras da linguagem e as dificuldades encontradas pela dupla.
\section{Linguagem Poor C}


\section{Exemplos}
Nos exemplos, buscamos identificar os diversos operadores, identificadores e demais características do input. Também prestamos atenção
aos blocos de comentários que foram problemáticos. 
\end{document}
