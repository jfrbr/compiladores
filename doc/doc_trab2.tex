\documentclass[a4paper,10pt]{article}
\usepackage[utf8]{inputenc}

%opening
\title{2º Trabalho de Compiladores - Analisador Sintático}
\author{André Siviero e Juan França}

\begin{document}

\maketitle

\begin{abstract}
Documentação referente à segunda parte do trabalho de compiladores, que consiste em um analisador sintático.
\end{abstract}

\section{Introdução}
O objetivo deste trabalho é estabelecer um analisador sintático para a linguagem PC (Poor C). O analisador sintático é a segunda parte de um compilador,
e sua função é analisar a sintaxe dos elementos contidos em um arquivo de entrada (código fonte). Assim, nesta segunda etapa, busca-se verificar se o arquivo de entrada
está dentro dos padrões da linguagem Poor C, que será definida mais abaixo.

Utilizando a ferramenta Bison para gerar um analisador sintático, a tarefa foi facilitada. Nesta documentação, são descritas as regras da linguagem e as dificuldades encontradas pela dupla.
\section{Linguagem Poor C}

Nesta seção iremos definir o que é a linguagem: algumas restrições e algumas vantagens de uso em comparação ao C.

\subsection{main}

Assim como em C, a linguagem PC também é executada a partir de uma main. A sintaxe é definida a seguir:

\begin{verbatim}
  int main{
    return 0; // return nao obrigatorio
  }
\end{verbatim}

Outras definições de main não irão funcionar (Por exemplo: void main() ).

\begin{verbatim}
 PROG: token_int_main token_abrep token_fechap token_abrec BLOCO token_fechac
;
\end{verbatim}

\subsection{Variáveis globais e funções}

Assim como na linguagem C, a linguagem PC também permite o uso de variavéis globais e outras funções. Porém, estas devem ser definidas
antes da main. Não existem declarações de funções.

A sintaxe para declarar é exatamente igual ao C. Nesse caso, mudamos o PROG acima definido:

\begin{verbatim}
 PROG:	BLOCO_GLOBAL token_int_main token_abrep token_fechap token_abrec BLOCO token_fechac
      | token_int_main token_abrep token_fechap token_abrec BLOCO token_fechac
;
\end{verbatim}
Onde,
\begin{verbatim}
BLOCO_GLOBAL: DECFUNC BLOCO_GLOBAL2
	      | DEC_VAR_GLOBAL BLOCO_GLOBAL2
;
BLOCO_GLOBAL2: /**/
	      | DECFUNC BLOCO_GLOBAL2
	      | DEC_VAR_GLOBAL BLOCO_GLOBAL2
		
;
DECFUNC: TIPO token_ident token_abrep PARAMETROS_TIPO token_fechap token_abrec BLOCO 
         token_fechac
         | TIPO token_ident token_abrep token_fechap token_abrec BLOCO token_fechac
;

DEC_VAR_GLOBAL: TIPO VAR DEC_VAR_GLOBAL2 token_ptevirgula
;

DEC_VAR_GLOBAL2: | token_virgula VAR DEC_VAR_GLOBAL2
;

TIPO: token_int | token_char | token_double | token_float | token_void
;
\end{verbatim}

A regra TIPO contém os tipos aceitos pela linguagem PC que são int, void, float, double e char. Pode-se ter funções com parâmetros e funções sem parâmetros.
Em ambas é necessário definir o tipo de retorno.\\

Em todos os casos do analisador sintático em que o nome da regra aparecer com um "2" após significa que aquele é a recursão da regra original.
Pode-se perceber que a declaração de funções e variáveis globais pode ser intercalada com este recurso, não limitando o programador
a declarar todas as funções utilizadas no programa para depois declarar todas as variáveis globais.

Agora vamos definir o que pode ser feito dentro de cada função.
\subsection{BLOCO}
Bloco é a regra onde está contido tudo o que pode ser feito dentro de uma função ou comando: atribuições, declarações, chamadas de funções etc.

\begin{verbatim}
 COMANDAO:   DEC_VAR token_ptevirgula
	| U_EXP_LIST token_ptevirgula
	| ATRIBUICAO token_ptevirgula
	| token_string token_ptevirgula
	| token_return U_EXP_LIST token_ptevirgula
	| token_return token_ptevirgula
	| token_ptevirgula
	| token_if token_abrep IF_EXP token_fechap COMANDAO token_else COMANDAO
	| token_if token_abrep IF_EXP token_fechap token_else COMANDAO
	| token_if token_abrep IF_EXP token_fechap token_abrec BLOCO token_fechac token_else 
                   token_abrec BLOCO token_fechac
	| token_if token_abrep IF_EXP token_fechap token_abrec BLOCO token_fechac token_else
                   COMANDAO
	| token_if token_abrep IF_EXP token_fechap COMANDAO token_else 
                   token_abrec BLOCO token_fechac
	| token_if token_abrep IF_EXP token_fechap token_else 
                   token_abrec BLOCO token_fechac
	| SWITCH
	| token_break token_ptevirgula
	| token_continue token_ptevirgula
	| LOOP	
;

BLOCO:	/**/ | COMANDAO BLOCO2 	| COMANDO BLOCO2
;

BLOCO2: /**/ | COMANDAO BLOCO2 | COMANDO BLOCO2
;
\end{verbatim}	

Comandão é o comando maior, definição de tudo o que pode ser feito na linguagem, com exceção dos comandos de seleção não-associados. 
Em Comandão também são definidos quais comandos necessitam de um ponto-e-vírgula ao final de sua chamada. Notadamente, comandos SWITCH,
LOOP e de seleção podem ter casos em que o ponto-e-vírgula não é necessário:

\begin{verbatim}
 if(condicao) {
  comandos...
 } 
 else 
  funcao;
 // ponto-e-virgula desnecessario 
\end{verbatim}


Bloco é usado para chamar na main e, como já dito, Bloco2 é a recursão do bloco.

\subsection{Declaração de variáveis}

Em cada função podem ser declaradas variavéis internas (pertencentes ao escopo da função), em qualquer lugar da função. Ou seja,
tem-se a liberdade de declarar uma variável, após usar um comando de seleção e depois declarar outra variável. Vetores e matrizes n-dimensionais
também são permitidos. Para definir o tamanho do vetor/matriz pode-se usar chamada de função, variáveis, etc. É permitido declação múltipla. Por outro lado,
não é permitido declararação junto com atribuição assim como no exemplo abaixo:

\begin{verbatim}
 int a = 3;
\end{verbatim}

Regras definidas no .y:

\begin{verbatim}
 DEC_VAR: TIPO VAR DEC_VAR2
;

DEC_VAR2: /**/ 
	| token_virgula VAR DEC_VAR2
;
PONTEIRO: token_vezes PONTEIRO2
;

PONTEIRO2: | token_vezes PONTEIRO2
;

VAR: PONTEIRO token_ident
	| token_ecom token_ident
	| token_ident
	| token_ecom token_ident COLCHETE
	| token_ident COLCHETE
	| PONTEIRO token_ident COLCHETE
;

COLCHETE : token_abrecol U_EXP_LIST token_fechacol COLCHETE2
;

COLCHETE2: /**/
      | token_abrecol U_EXP_LIST token_fechacol COLCHETE2
      ;

\end{verbatim}

A regra U\underline{ }EXP\underline{ }LIST será definida mais abaixo, e corresponde às expressões válidas.
\subsection{Atribuição}

Nesta seção serão definidos os comandos de atribuição:

\begin{verbatim}
 = , +=, -=, *=, /=, &=, ^=, |=, >>=, <<=
\end{verbatim}

Na PC, os comandos ++ e -- não são considerados uma atribuição, estando definidos como expressão. Não é permitida atribuição múltipla, por exemplo, a = b = 1. A seguir, segue como está no .y:

\begin{verbatim}

 ATRIBUICAO: VAR token_igual TO_ATRIB
	  | VAR token_maisigual TO_ATRIB
	  | VAR token_menosigual TO_ATRIB
	  | VAR token_vezesigual TO_ATRIB
	  | VAR token_divisaoigual TO_ATRIB
	  | VAR token_ecomigual TO_ATRIB
	  | VAR token_xnorigual TO_ATRIB
	  | VAR token_ouigual TO_ATRIB
	  | VAR token_shiftdireitaigual TO_ATRIB
	  | VAR token_shiftesquerdaigual TO_ATRIB
;

TO_ATRIB:  U_EXP_LIST | token_string;

\end{verbatim}

A regra TOATRIB é a parte direita da atribuição. Observe que podemos atribuir uma expressão a uma variável. Na próxima seção definiremos o que é uma expressão na PC.
Outra observação importante: no léxico, o número (inteiro ou float) não pode ser negativo, portanto inserimos o caso em que ele pode ser negativo no sintático.

\subsection{Expressão}
\begin{verbatim}
U_EXP: EXP token_igualigual EXP
      | EXP token_maior EXP
      | EXP token_menor EXP
      | EXP token_maiorigual EXP
      | EXP token_menorigual EXP
      | EXP token_diferente EXP
      | EXP
;

IF_EXP :
      U_EXP_LIST
;

U_EXP_LIST : U_EXP U_EXP_LIST2
;

U_EXP_LIST2 : | token_ecomecom U_EXP U_EXP_LIST2 | token_ou U_EXP U_EXP_LIST2
;

EXP: EXP token_mais TERMO
    | EXP token_menos TERMO
    | TERMO
;

TERMO: TERMO token_vezes FATOR
      | TERMO token_divisao FATOR
      | TERMO token_mod FATOR
      | FATOR
;

FATOR: token_num_float 
	  | token_num_inteiro 
	  | VAR | token_abrep U_EXP_LIST token_fechap 
	  | token_letra 
	  | CHAMADA_FUNCAO
	  | VAR token_maismais
	  | token_maismais VAR
	  | token_menosmenos VAR
	  | VAR token_menosmenos
	  | token_menos token_num_float
	  | token_menos VAR
	  | token_menos token_num_inteiro
;

\end{verbatim}

U\underline{ }EXP são todas as expressões (tanto booleanas quanto numéricas). A precedência de operadores foi resolvida utilizando regras auxiliares (TERMO E FATOR).
O caso de condicionais múltiplas é permitido, por exemplo, if (a $>$ 3 \&\& a $<$ 10).

\subsection{Chamada de função}

As chamadas de função são simples. Implementadas do seguinte modo:

\begin{verbatim}

CHAMADA_FUNCAO : token_ident token_abrep PARAMETROS token_fechap
		  | token_ident token_abrep token_fechap
;

PARAMETROS: U_EXP_LIST PAR2 | token_string PAR2;

PAR2: | token_virgula U_EXP_LIST PAR2 | token_virgula token_string PAR2;

\end{verbatim}

\subsection{Comandos de seleção}

Definiremos agora os comandos de seleção da linguagem PC.

\subsubsection{if-else}

Estes são bem complicados e foram a parte mais complicada do trabalho por conta das diversas configurações e problemas com precedência.
Comandos associados (possuindo else) foram definidos no próprio bloco, com as diversas configurações de possibilidade de chaves. Para
evitar problemas com shift/reduce, os comandos nao associados (sem else) são definidos em separado pelas seguintes regras:



\begin{verbatim}

COMANDO: CMD_NAO_ASSOC | CMD_NAO_ASSOC_CHAVE
;

CMD_NAO_ASSOC_CHAVE : token_if token_abrep IF_EXP token_fechap token_abrec BLOCO token_fechac
		      | token_if token_abrep IF_EXP token_fechap token_abrec BLOCO token_fechac token_else COMANDO
;

CMD_NAO_ASSOC : token_if token_abrep IF_EXP token_fechap COMANDAO
		| token_if token_abrep IF_EXP token_fechap COMANDO
		| token_if token_abrep IF_EXP token_fechap COMANDAO token_else COMANDO
;

\end{verbatim}

Estas regras seguem o que foi definido nos slides em sala de aula.
\subsubsection{switch}

O parâmetro de teste do switch pode ser um char ou um número inteiro. É obrigatório ter uma cláusula default. A definição segue a seguir:

\begin{verbatim}

SWITCH: token_switch token_abrep VAR token_fechap token_abrec SWITCH_BLOCK token_fechac
;

SWITCH_BLOCK : token_default token_doispontos BLOCO 
		| token_case token_num_inteiro token_doispontos BLOCO SWITCH_BLOCK2 token_default token_doispontos BLOCO
		| token_case token_letra token_doispontos BLOCO SWITCH_BLOCK2 token_default token_doispontos BLOCO
;
SWITCH_BLOCK2 : 
		| token_case token_num_inteiro token_doispontos BLOCO SWITCH_BLOCK2
		| token_case token_letra token_doispontos BLOCO SWITCH_BLOCK2 
;
\end{verbatim}

\subsection{Comandos de repetição}
A seguir definiremos os comandos de repetição: FOR, WHILE e DO WHILE assim como na linguagem C. A regra abrange todos:
\begin{verbatim}
 LOOP : FOR_LOOP | DO_WHILE_LOOP | WHILE_LOOP ;
\end{verbatim}

\subsubsection{for}

O comando FOR é formado da seguinte forma: for (lista de atribuicoes; condicao; lista de comandos). Pode-se ter várias atribuições dentro dos parenteses do for,
a condição pode ser qualquer expressão (numérica ou booleana) e pode ser inserido qualquer comando a ser executado em cada iteração.

As diferentes atribuições e comandos são separados por vírgulas.

\begin{verbatim}

FOR_LOOP : token_for token_abrep ATRIBUICAO_LIST token_ptevirgula U_EXP_LIST token_ptevirgula COMMAND_LIST token_fechap COMANDAO
	   | token_for token_abrep ATRIBUICAO_LIST token_ptevirgula U_EXP_LIST token_ptevirgula COMMAND_LIST token_fechap token_abrec BLOCO token_fechac
	   | token_for token_abrep ATRIBUICAO_LIST token_ptevirgula token_ptevirgula COMMAND_LIST token_fechap COMANDAO
	   | token_for token_abrep ATRIBUICAO_LIST token_ptevirgula token_ptevirgula COMMAND_LIST token_fechap token_abrec BLOCO token_fechac
;

ATRIBUICAO_LIST : | ATRIBUICAO ATRIBUICAO_LIST2
;

ATRIBUICAO_LIST2 : | token_virgula ATRIBUICAO ATRIBUICAO_LIST2
;

COMMAND_LIST : | ATRIBUICAO COMMAND_LIST2 | EXP COMMAND_LIST2
;

COMMAND_LIST2 : | token_virgula ATRIBUICAO COMMAND_LIST2 | token_virgula  EXP COMMAND_LIST2
;

\end{verbatim}

\subsubsection{while}

O comando while é parecido com o de C: while (condição). Enquanto esta condição resultar em um valor diferente de zero. A condição pode ser qualquer expressão.
Para executar em loop pode haver comandos na forma de um bloco englobado por chaves ou um comando único como definido em Comandão.

\begin{verbatim}
 WHILE_LOOP: token_while token_abrep U_EXP_LIST token_fechap COMANDAO
	    | token_while token_abrep U_EXP_LIST token_fechap token_abrec BLOCO token_fechac
;
\end{verbatim}

\subsubsection{do while}

Parecido com while e sob mesmas condições, porém o que deve ser executado vem antes da condição.

\begin{verbatim}
DO_WHILE_LOOP : token_do COMANDAO token_while token_abrep U_EXP_LIST token_fechap token_ptevirgula
		| token_do token_abrec BLOCO token_fechac token_while token_abrep U_EXP_LIST token_fechap token_ptevirgula
;

\end{verbatim}

\section{Exclusões}
Na PC (fazendo jus ao nome), não tem structs, nem DEFINEs, não tem typedef e também não pode incluir bibliotecas. O acesso a ponteiros
por $->$ também não é permitido.

\section{Exemplos}
Nos exemplos, exploramos diversas situações para os diversos tipos de usuários. Foram criadas 10 situações, 5 situações corretas e 5 
situações erradas. Um teste com situações bizarras e mais difíceis também foi incluído.
\end{document}
