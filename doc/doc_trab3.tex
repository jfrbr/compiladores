\documentclass[a4paper,10pt]{article}
\usepackage[utf8]{inputenc}

%opening
\title{3º Trabalho de Compiladores - Analisador Sintático}
\author{André Siviero e Juan França}

\begin{document}

\maketitle

\begin{abstract}
Documentação referente à terceira parte do trabalho de compiladores, que consiste em um analisador semântico.
\end{abstract}

\section{Introdução}
O objetivo deste trabalho é estabelecer um analisador semântico para a linguagem PC (Poor C). O analisador semântico é a terceira parte de um compilador,
e sua função é analisar a coerência dos elementos contidos em um arquivo de entrada (código fonte). 

Utilizando a ferramenta Bison para gerar um analisador sintático, a tarefa foi facilitada. Nesta documentação, são descritas as regras da linguagem e as dificuldades encontradas pela dupla.
\section{}



\section{Exemplos}
Nos exemplos, exploramos diversas situações para os diversos tipos de usuários. Foram criadas 10 situações, 5 situações corretas e 5 
situações erradas. Um teste com situações bizarras e mais difíceis também foi incluído.
\end{document}
